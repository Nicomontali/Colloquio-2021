\section{Claude Lévi-Strauss lettore di Marcel Mauss}
Potrebbe sembrare un torto nei confronti della storia delle scienze umane anteporre l'interpretazione levistraussiana di Marcel Mauss data da Lévi-Strauss alla rilettura delle tesi di Émile Durkheim, zio del primo e nome d'importanza capitale nella storia della sociologia e delle scienze umane francesi, tuttavia, ciò si giustifica dal momento che la pubblicazione dell'\textit{Introduzione} alle opere di Marcel Mauss da parte di Lévi-Strauss stesso avviene nel 1950, mentre le tesi esposte ne \textit{Il totemismo oggi}, facenti riferimento a Durkheim (tra gli altri), pervengono all'attenzione del grande pubblico solo nel 1962. Si proceda quindi con questa scansione.

È il 1950 quando una raccolta antologica di saggi di Marcel Mauss viene pubblicata dalla Presse Universitarie de France di Parigi. Il volume, originariamente intitolato \textit{Sociologie et anthropologie}, è edito in Italia da Einaudi ben quindici anni dopo, nel 1965. 
L'opera, composta da saggi pubblicati autonomamente sulle pagine di riviste avvalendosi della collaborazione dell'amico e collega Henri Hubert, può considerarsi a buon diritto come una \textit{summa} del pensiero di Mauss, tanto prodigo nelle intuizioni, straordinariamente moderne per il suo tempo, quanto avaro nella produzione scritta, per lo meno rispetto allo zio Émile Durkheim.
L'edizione italiana, con una prefazione di Ernesto De Martino, vede mantenuta e tradotta integralmente l'introduzione di Claude Levi-Strauss\footnote{\cite{mauss1965teoria} [pp. \textsc{xv-liv}]}, introduzione che peraltro ha ricevuto approvazione e notorietà in terra americana, portando, nel 1987, la casa editrice Taylor \& Francis a pubblicarla come volume autonomo\footnote{\cite{levi1987introduction}}. Percorrendo le parole di Levi-Strauss, tuttavia, presto ci si accorge che quella che si ha tra le mani è ben più di un'introduzione all'opera di Mauss: il lavoro di quest'ultimo, infatti, diventa quasi un pretesto, la materia prima di una rielaborazione estremamente originale.

RICORDATI DI AGGIUNGERE LE COSE DA KARSENTI L'UOMO TOTALE \cite{karsenti1997uomo}

\section{Claude Lévi-Strauss lettore di Émile Durkheim}
Il volume pubblicato nel 1962, \textit{Il totemismo oggi}\footnote{\cite{levi2020totemismo}.}, costituisce la prima parte di un progetto più ambizioso, che vedrà il suo compimento nel volume \textit{Il pensiero selvaggio}\footnote{\cite{levi2010pensiero}}, il quale sarà oggetto di osservazione approfondita nel capitolo successivo.
\textit{La pensée sauvage}, titolo che in lingua francese rimanda esplicitamente al fiore \textit{Viola tricolor} (Viola del pensiero in lingua italiana), va a comporre con \textit{Le totémisme aujourd'houi} un unico progetto: ritrovare all'interno delle istituzioni religiose e sociali le caratteristiche di un sistema di classificazione.
Il volume \textit{Il totemismo oggi}, più breve del suo "seguito", è composto con uno stile argomentativo rigoroso, pieno di dimostrazioni e riferimenti accademici eruditi, e ha il compito di mostrare la consapevolezza storica con cui Lévi Strauss si inserisce nel dibattito sul totemismo.
I principali interlocutori accademici con cui Lévi-Strauss deve confrontarsi all'interno dell'opera sono altri antropologi che all'epoca già avevano tentato di spiegare il fenomeno del totemismo senza successo: nonostante il problema fosse già stato sondato più volte all'epoca, nessuno era riuscito fino a quel momento a produrre un'interpretazione esaustiva, in grado di spiegare il fenomeno totemico in tutti i suoi aspetti. Del resto, è da sottolineare, il fenomeno del totemismo aveva attirato l'attenzione di intellettuali di diversa formazione, tra i quali spicca Henri Bergson.
Nonostante la polemica presa di distanza che Lévi-Strauss conduce nella prima parte del celeberrimo \textit{Tristi Tropici}\footnote{\cite{levi1960tristi}, pp.49-52.}, la formazione in materia che ricevette influenzò profondamente il modo di condurre la ricerca etnologica. Una graffiante pagina dell'opera sopracitata accusa ferocemente la filosofia di essersi ridotta a un vuoto gioco di retorica, costituito da elementi evocati con la parola e dissolti da questa stessa\footnote{Ho cominciato allora a capire che tutti i problemi, gravi o futili, possono essere liquidati applicando un metodo sempre identico, che consiste nel contrapporre due punti di vista tradizionali sulla questione,; introdurre cioè il primo con le giustificazioni del senso comune, per distruggerlo poi con il secondo; infine rigettarli uno da una parte e uno dall'altra, adottando invece un terzo punto di vista che riveli il carattere ugualmente parziale dei due altri, ricondotti con artifici di vocabolario agli aspetti complementari di una stessa realtà: forma e sostanza, contenente e contenuto, essere e parere, continuo e discontinuo, essenza ed esistenza ecc. [\cite{levi1960tristi}, p.49]}