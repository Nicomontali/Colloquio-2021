Per capire appieno la profondità del dibattito che si svolse a metà del XX secolo tra Claude Levi-Strauss e Jean-Paul Sartre occorrerebbe molto più di un lavoro di colloquio, dal momento che si tratta di due delle più influenti e originali personalità intellettuali in Francia dell'epoca. Per svolgere nella maniera più chiara possibile una ricostruzione storico-filosofica di tale dibattito, pertanto, è mia intenzione utilizzare le opere centrali dello scambio tra i due intellettuali - si perdoni il paragone botanico - come il tronco di un albero, dal quale discendere a ritroso nel tempo fino alle radici e, parimenti, procedere in avanti nella storia della filosofia fino a concentrarsi su alcuni rami. Ovviamente, per circoscrivere il lavoro da svolgere, l’attenzione dell’autore si concentrerà sulle opere di pensatori e intellettuali più direttamente influenti sugli autori analizzati ed influenzati da questi, ovvero, per tornare a fare uso del paragone botanico, sui rami principali e più prossimi al tronco, e sulle radici meno in profondità nel terreno.
Il lavoro, pertanto, si articolerà in tre capitoli per comodità cronologica, anche se il primo e il terzo capitolo costituiranno due metà di un unico discorso.
Ricalcando il saggio di Bruno Karsenti all'interno del recente \textit{Simposio L\'evi-Strauss. Uno sguardo dall'oggi}\footnote{\textit{Le proprietà di uno specchio deformante} in \cite{simposio2013}}, il compito di Jean-Paul Sartre e della sua opera, in particolare i due volumi componenti la monumentale \textit{Critica della Ragion dialettica}\footnote{\cite{sartre1960critique}. \cite{sartre1984critica}} sarà quello di fungere da specchio deformante, ossia evidenziare le peculiarità della posizione di Lévi-Strauss attraverso l'adozione di una concezione della ragione diametralmente differente, anche se non per questo incompatibile.
